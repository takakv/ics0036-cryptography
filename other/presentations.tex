\documentclass[usegeometry,parskip=half]{scrartcl}

\usepackage[
  a4paper,
  top=25mm,
  bottom=25mm,
  left=25mm,
  right=25mm,
  footskip=10mm
]{geometry}

\usepackage{csquotes}
\setquotestyle{british}

\usepackage[hidelinks]{hyperref}
\pagenumbering{gobble}

\usepackage{microtype}
\usepackage{fontspec}
\usepackage{unicode-math}

\setmainfont{Stix Two Text}
\setsansfont{TeX Gyre Heros}
\setmathfont{Stix Two Math} 

\begin{document}
\textbf{ICS0036 Cryptography --- group presentations}

Your group, group presentation date, and topic must be finalised by 31.10.2025 23:59 EET.
If needed, you can change the topic later by contacting me.

10\% of your final grade comes from the group presentation.
The group presentations are in-person, and all group members must participate.

There are three possible presentation days, with up to 6 groups presenting each day.
Groups shall be no larger than 3 people.
Groups may be smaller as long as there are no more than 6 groups presenting each of the three days.


If you know that you cannot attend \emph{any} of the group presentation dates, please tell me as soon as possible.
In this case, you will prepare your presentation alone, and we will find a date for a presentation over Teams.

You will have 10 minutes for your presentation.
I am strict with timing, and will stop you after 11 minutes at most.
Each member must present, with presentation time roughly equally divided between members, e.g. every member presents a 3-minute slice.
However, your presentation can be shorter than 10 minutes if you can get all the core information across more succinctly.

After the presentation, there will be 4-5 minutes for questions: at minimum, I will ask each group member one question about the slice they presented.
Final scoring can therefore vary by member depending on the demonstrated familiarity with the topic.

How you organise your research is up to you.
You can research everything together and then divide the presentation, or everyone can search and present their own part.
The final presentation should still be cohesive, with no repetition.

If you have issues with a group member (e.g. they do no work), let me know as soon as possible.
It is not your job to carry a weak group member.
We can then discuss what to do, but it is fine that you plan a 6-minute presentation and let the third member suffer the consequences of their inaction.
TL;DR I will not let a slacking teammate penalise your work if you let me know about it.

What am I looking for in the presentation?

I have not written out exact requirements for two reasons:
\begin{itemize}
  \item They highly depend on your topic choice
  \item I wish to see what you consider important
\end{itemize}

I have only referenced a potential starting point for each of the topics.
You are free to find and use your own sources.

The rules of thumb are:
\begin{itemize}
  \item If analysing an attack:
  
  What?, When?, How?, Impact/severity (can also be perceived), Mitigation, Lessons learnt

  \item If analysing a protocol:
  
  Design goal/protocol purpose, brief protocol overview, role of different components
\end{itemize}

You can approach the task in this way: you have to explain the gist of an attack or of a protocol to the CTO (Chief Technical Officer) of your company.
The CTO is not a cryptographer, so you must explain anything more complex than what encryption/signing/hashing is.
It is not enough to simply \enquote{name drop} stuff, e.g. if you mention a padding oracle, you also then have to explain it.

If you have doubts about the expectations or the direction you are going in, ask me.

\newpage

Topics:
\begin{itemize}
  \item Double ratchets --- how do they work?

  \url{https://signal.org/docs/specifications/doubleratchet/}

  \item MTProto --- how does Telegram's protocol work?
  
  \url{https://core.telegram.org/mtproto}

  \item \enquote{Formally verified} TLS 1.3 --- how it came to be
  
  \url{https://tamarin-tls.cispa.de/en/deceptive-security.html}

  \item Formal verification in cryptography --- tooling and purpose
  
  \url{https://crypto.stackexchange.com/a/34326}

  \item S/MIME and its purpose
  
  \url{https://www.globalsign.com/en/blog/what-is-s-mime}

  \item Blind signatures and their use cases
  
  \url{https://www.hit.bme.hu/~buttyan/courses/BMEVIHIM219/2009/Chaum.BlindSigForPayment.1982.PDF}\\
  \url{https://csrc.nist.gov/csrc/media/presentations/2022/stppa4-blind-sigs/images-media/20221121-stppa4--julian-loss--blind-signatures.pdf}

  \item SHA-3 reference implementation bug
  
  \url{https://eprint.iacr.org/2023/331}

  \item Heartbleed
  
  \url{https://www.heartbleed.com}

  \item KRACK attacks
  
  \url{https://www.krackattacks.com}

  \item Sony ECDSA fail
  
  \url{https://youtu.be/84WI-jSgNMQ}

  \item Wii signing bug
  
  \url{https://wiibrew.org/wiki/Signing_bug}

  \item Apple goto fail
  
  \url{https://gotofail.com}

  \item Logjam
  
  \url{https://weakdh.org/}

  \item Bar Mitzvah attack
  
  \url{https://www.blackhat.com/docs/asia-15/materials/asia-15-Mantin-Bar-Mitzvah-Attack-Breaking-SSL-With-13-Year-Old-RC4-Weakness-wp.pdf}

  \item \enquote{Original} Bleichenbacher's PKCS\#1 v1.5 attack
  
  \url{https://crypto.stackexchange.com/a/12706}

  \item ROBOT
  
  \url{https://robotattack.org}

  \item BEAST
  
  \url{https://vnhacker.blogspot.com/2011/09/beast.html}

  \item POODLE
  
  \url{https://nvd.nist.gov/vuln/detail/CVE-2014-3566}

  \item DUHK
  
  \url{https://duhkattack.com}

  \item CBC padding oracle
  
  \url{https://link.springer.com/chapter/10.1007/3-540-46035-7_35}\\
  \url{https://blog.cloudflare.com/padding-oracles-and-the-decline-of-cbc-mode-ciphersuites/}

  \item CVE-2008-0166 and its impact
  
  \url{https://nvd.nist.gov/vuln/detail/cve-2008-0166}\\
  \url{https://16years.secvuln.info}

  \item MemJam
  
  \url{https://arxiv.org/abs/1711.08002}

  \item Rowhammer on Bitcoin
  
  \url{https://github.com/demining/Rowhammer-Attack}

  \item TLBleed
  
  \url{https://www.vusec.net/projects/tlbleed/}

  \item CurveSwap
  
  \url{https://eprint.iacr.org/2018/298}
\end{itemize}

\end{document}
