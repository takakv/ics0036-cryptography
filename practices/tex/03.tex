\documentclass{practice}

\usepackage{listings}

\lstset{
basicstyle=\small\ttfamily,
columns=flexible,
breaklines=true
}

\begin{document}
\begin{center}
Practice session 3 --- 17.09.25
\end{center}

\textbf{Task 1 --- OpenSSL}

OpenSSL\footnote{\url{https://www.openssl.org}} is a popular cryptographic tool/library which is perhaps best known in the context of TLS.
It offers a wide variety of functionalities and is used not only by cryptographers, but also developers and sysadmins.

OpenSSL has however had its fair share of weaknesses, and I personally do not like it very much: it can be fairly tricky to use and it is easy to make mistakes.
As such, I typically prefer other tools (many of which are derived from OpenSSL), but you definitely need to know that OpenSSL exists.

For this task, install OpenSSL.

\textbf{Task 2 --- Concatenation and padding}

For this task, implement the three non-deprecated padding schemes that we saw in class.
Split a long byte string into blocks, and pad it if necessary.

\textbf{Task 3 --- AES in Python}

For this task, create a python program which uses AES-CBC to encrypt and decrypt files on the disk. Make sure that the IV is randomly generated when not explicitly provided.
Make sure that block sizes are consistent: pad messages when needed.

I personally like PyCryptodome\footnote{\url{https://www.pycryptodome.org}} but there are some other good libraries as well.

Some useful links:
\begin{itemize}
  \item \url{https://www.pycryptodome.org/src/cipher/aes}
  \item \url{https://www.pycryptodome.org/src/util/util}
\end{itemize}

\textbf{Task 4 --- AES test vectors}

During the lectures, we learnt that AES is standardised by NIST.

NIST has a program called the `Cryptographic Algorithm Validation Program' (CAVP) and part of it are testing specifications and \emph{test vectors} for block ciphers.

Fetch some test vectors from their website, and use them to test the AES implementation.

\url{https://csrc.nist.gov/projects/cryptographic-algorithm-validation-program/block-ciphers}

\textbf{Task 5 --- Testing with OpenSSL}

With OpenSSL, you can encrypt using 128-bit AES-CBC with the command:
\begin{verbatim}
openssl enc -aes-128-cbc -K <keyhex> \
    -in <infile> -out <outfile> -iv <ivhex>
\end{verbatim}
and decrypt with:
\begin{verbatim}
openssl enc -d -aes-128-cbc -K <keyhex> \
    -in <infile> -out <outfile> -iv <ivhex>
\end{verbatim}

Use the \texttt{-nopad} option if needed.

Test the AES implementation with OpenSSL.

\textbf{Task 6 --- Error propagation \& modes of operation}

Try to see what happens to the ciphertext when you modify different parts of the plaintext, e.g. how does the ciphertext change when you change a single bit in the second plaintext block CBC mode? What about changing a bit in the first plaintext block in ECB mode?

What happens with decryption when you make a change in a ciphertext block?
What happens if you drop/lose a ciphertext block in AES-CBC?

Try to reason first using the diagram and then confirm in code.

\textbf{Task 7 --- Key reuse, IV reuse}

Try out the following scenarios and compare the outputs.
\begin{enumerate}
  \item Encrypt the same data twice with the same key and same IV.
  \item Encrypt the same data twice with the same key and different IV.
  \item Encrypt the same data twice with a different key but the same IV.
\end{enumerate}

Try this with different modes of operation.
Would you be able to write a distinguisher?

\textbf{Task 8 --- The cursed task}

Hopefully by now you have a solid understanding of CBC mode.
By using a library implementation of AES-ECB, build AES-CBC from it.

\end{document}
