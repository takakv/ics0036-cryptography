\documentclass{practice}

\usepackage{crysymb}

\title{14}
\date{\DTMdisplaydate{2024}{12}{05}{4}}

\begin{document}
\maketitle

\begin{task}{Shamir}
  Please read through the Shamir PDF that I uploaded to Moodle.

  I unfortunately do not have the time to prepare a template for practical tasks with it.
\end{task}

\begin{task}{RSA vote verification}
  \textit{Background.}
  The IVXV voting system used for Estonian i-voting since 2017 uses the ElGamal cryptosystem for encrypting votes.
  The homomorphic nature of ElGamal enables two important mechanisms:
  \begin{enumerate}
    \item Zero-knowledge proofs of correct decryption,
    \item Re-encryption mixnet.
  \end{enumerate}
  Prior to IVXV, RSA-OAEP was used as the election cryptosystem.
  Because RSA-OAEP does not have homomorphic properties, these two mechanisms were not possible.

  The verification application was introduced in 2013, when RSA-OAEP was still used.
  We saw some weeks ago that due to the properties of the ElGamal cryptosystem, knowing the encryption randomness would allow to decrypt a particular ciphertext.
  No such property exists for RSA-OAEP, where the data being encrypted is the message combined with the randomness.
  Still, somehow, when given the encryption randomness, the verification application was able to display to the voter the contents of the encrypted ballot.

  \textit{Setting.}
  In the file \texttt{rsa\_vote.py} is some data which simulates an election cryptosystem.
  The election public key and the list of candidates are provided in the file.
  The voting application encryption function is also provided in the file.

  The file includes a ciphertext and the random value used in the encryption process.
  You can assume that you have somehow obtained these values by placing a trojan on my smartphone which I used to verify my vote.

  \textit{Task.}
  Determine who my favourite \enquote{Haikyū!!} character is by making use of the leaked verification information.
  Submit the character name on Moodle along with a few lines of explanation how you obtained it.
  No need to upload your code.

  \begin{tcolorbox}[title=NB!]
    In practice, the verification application verifies many things, including the signature on the ciphertext and the vote qualifying elements.
    However, it is the user's responsibility to alert the organisers if the verification application displays a choice other than the intended one, or if it warns that some check has failed.
  \end{tcolorbox}
\end{task}

\end{document}
