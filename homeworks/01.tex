\documentclass{homework}

\usepackage{enumitem}
\usepackage{tcolorbox}

\usepackage{tikz}
\usetikzlibrary{positioning,calc}

\newcommand*{\ENC}{\mathsf{Enc}}

\usepackage{listings}

\lstset{
basicstyle=\small\ttfamily,
columns=flexible,
breaklines=true
}

\usepackage{fancyvrb}
\fvset{listparameters=\setlength{\topsep}{0pt}\setlength{\partopsep}{0pt}}

\title{01}
\author{Taaniel Kraavi}
\date{\DTMdate{2025-10-04}}

\begin{document}
\maketitle

\textbf{Organisational details}

\begin{itemize}
  \item Include in your report approximately how long the homework took for you.
  \item I will give you feedback on Discord.
  Check it after receiving your (preliminary) grade.
  \item Read through task descriptions carefully.
  \item You may request deadline extensions until 01.10.25 23:59 EEST.
  \item You can submit up to one week after the deadline with a 50\% point penalty.
  0 afterwards.
  \item Grace clause:
  \begin{itemize}
    \item If you submit your assignment by the deadline, you have the opportunity to fix (some) mistakes I point out to gain back up to half of the lost points.
    \item You cannot get any points back for a task that is entirely missing or broken, e.g. code that does not run at all or does nothing.
    If your code does not pass the assertions/test script (if available), include that info in your report.
    Otherwise I will consider your code to be broken, i.e. no grace.
    Test your code!
    \item The grace clause does not apply if you requested an extension or missed the original deadline.
  \end{itemize}
  \item You may share hints with each-other, but no answers/code.
  Ask questions in the server if they could be useful for everyone.
  \item You may---and I encourage you to---ask me questions (preferably during practices).
  \item You may use AI as help, but not to solve the tasks themselves.
  \begin{itemize}
    \item Acceptable: e.g. checking your understanding, asking how to loop in Python
    \item Not acceptable: e.g. generating any code logic or cryptographic operations
    \item If you use AI in any form, include in your report what you used it for
    \item I recommend that you ask in the server/me instead of using AI
  \end{itemize}
  \item If I catch you plagiarising, copying, or using AI for code logic, I will report you to the faculty and fail you for the course.
  \begin{itemize}
    \item If during the midterm/exam you completely fumble stuff that you got points for in homework tasks, you will have to orally explain why it is so.
    \item If you do not convince me of homework authorship, I will report you for cheating and fail you.
  \end{itemize}
\end{itemize}

\newpage

\begin{center}
  Theory tasks
\end{center}

\textbf{Additional instructions}

\begin{itemize}
  \item Submit as a PDF file called \texttt{studentid-01.pdf} where \enquote{studentid} is your 6 letter student ID/username (\enquote{takraa}) for me.
  \item Submit it on Moodle as a standalone file (not as part of the code ZIP).
  \item For a second submission, name your new file \texttt{studentid-01.2.pdf}.
\end{itemize}

\begin{task}{A little about you}
  What role are you currently working as/what would you like to work as?
  Is or do you anticipate cryptography to be somehow related to your work?
  Was there any specific reason (other than the 6 ECTS) that made you pick this course/what do you hope to gain from it?

  \textit{Just a few sentences (3--4) is fine.}
\end{task}

\begin{task}{Knowing your system}
  What operating system do you use? Give three examples of disk-encryption software supported on your OS (include links).

  Do you use full disk encryption? If not, would you start using it?
\end{task}

\begin{task}{Knowing your tools}
  What programming language do you use most/are you most comfortable in?
  For this language, list the following:
  \begin{enumerate}
    \item Standard library/namespace/module (if any) for getting cryptographic randomness.
    \item Standard library/namespace/module (if any) for encryption.
    \item Three well regarded 3\textsuperscript{rd} party libraries for cryptographic stuff (any kind).
    \item Some cool cryptographic library/tool/project (not already named above).
  \end{enumerate}

  Include your sources!
\end{task}

\begin{task}{Cryptographer's Rosetta}
  If your mother tongue is not English, try to find a frequency diagram for your language (as we did for the shift cipher).

  If your mother tongue is English, or you cannot find a diagram for your mother tongue, find a frequency diagram for bigrams (two letter pairs, e.g. \enquote{at}, \enquote{to}, \enquote{no}, \dots) or trigrams.

  Include the source!
\end{task}

\begin{task}{Knowing your processor}
  Benchmark how long it takes for your computer to iterate over the 32-bit space (practice 1). How long would it take for your computer to cover the 128-bit space based on this info? Show your calculations.
\end{task}

\begin{task}{A pledge}
  \enquote{%
    I understand that I should not implement my own cryptography for any production system, i.e. for any purpose other than a personal learning challenge.
    I understand also that any implementations done during this course are only for learning purposes, and are not for real world use. I understand that due diligence is crucial.%
  }

  \textit{Answer with yes/no.}
\end{task}

\newpage
\setcounter{task}{0}

\begin{center}
  Practical tasks
\end{center}

\textbf{Additional instructions}

\begin{itemize}
  \item I must be able to run your program with Python 3.13 and OpenSSL v3.5.1\footnotemark{}.
  \footnotetext{You may use older versions of Python and OpenSSL as long as you do not use deprecated functionality.}
  \item Do not import any 3\textsuperscript{rd} party module\footnotemark{} not specified in the template files.
  \footnotetext{You may change \texttt{Crypto} to \texttt{Cryptodome} if needed.}
  You are free to import any built-in module.
  If you feel the need to import some other 3\textsuperscript{rd} party module, please check with me beforehand!
  \item Your program must not have any unhandled errors that are dependent on user provided data.
  Print a message and exit with an error code without printing the trace when you encounter a situation that should error.
  However, you need not handle file-based errors, e.g. file read/write errors, although it is good practice to handle them.
  \item Submit on Moodle as a ZIP file called \texttt{code.zip}.
  \item Your Python files must have the same name as the template files but without the \texttt{template} suffix, e.g. \texttt{chacha20\_template.py} becomes \texttt{chacha20.py}.
  \item You may change everything in the template files as long as I can run your program with the CLI arguments specified in the templates.
  \item Your ZIP must have the following structure/contents:
  \begin{Verbatim}
code/
├── aes_ecb_cbc.py
├── chacha20.py
├── key.txt
├── decryption.txt
└── ciphertexts/
    ├── ct-1.bin
    ├── ct-2.bin
    ├── ct-3.bin
    ├── ct-4.bin
    └── ct-5.bin
  \end{Verbatim}
\end{itemize}

\begin{task}{Cryptohack}
  Create an account on \url{https://cryptohack.org/courses/}, and try to finish at least the 10 courses of the \enquote{Introduction to Cryptohack}.
  If you cannot finish the 10 courses, write in your report what tasks you were stuck with and what did you try.

  Include a screenshot of your module completion and your username in the PDF report.
  Include also the full link to your user profile, e.g. \url{https://cryptohack.org/user/ch-username}.

  I do recommend that you to attempt the \enquote{Symmetric Cryptography} module as well, but that is not a mandatory part of the homework.
\end{task}

\begin{task}{AES CBC from ECB}
  Homework template file: \texttt{aes\_ecb\_cbc\_template.py}

  The task is to use a library implementation of AES (you need not implement AES yourself) in ECB mode, and build the CBC mode out of it.
  That is, you must instantiate AES with
  \begin{center}
    \texttt{aes = AES.new(key, AES.MODE\_ECB)}
  \end{center}
  Use the PKCS\#7 padding for messages that do not divide into blocks.
  You must implement the pad/unpad functions yourself.
  \begin{tcolorbox}[title=Padding]
    Recall that padding is always added (at least for PKCS) even if the message length is a multiple of the block size.
    If the message length is a multiple of the block size, add an entire block of just padding.
  \end{tcolorbox}

  Your program must accept the following CLI arguments (handled by the template):
  \begin{itemize}
    \item \texttt{-e}/\texttt{-d}: whether to encrypt or decrypt the input data
    \item \texttt{-i}: the input filename
    \item \texttt{-o}: the output filename
    \item \texttt{--key}: the key (a string of hex digits)
    \item \texttt{--iv}: the iv (a string of hex digits)
  \end{itemize}

  For encryption, providing the IV on the CLI must be optional.
  If the IV is provided, the program must use it.
  If not, it must generate a cryptographically random IV.

  \begin{tcolorbox}[title=Note]
    In practice, providing secrets (e.g. keys, passwords, etc) directly on the CLI as an argument to a program is a terrible idea, since they typically get logged into your terminal history!
  \end{tcolorbox}

  Additional program requirements:
  \begin{itemize}
    \item It must work with any key-size supported by AES.
    \item It must encrypt/decrypt the provided data and output the result to a file.
    \item The files must be read and written to as bytes on disk.
    \item It should print the used key and IV to the CLI as hex strings.
    \item It should not crash if a dubious key/IV is provided (or other edge cases): handle your errors! (I/O error handling is not necessary)
    \item You must handle errors proactively: simply catching errors thrown by PyCryptodome will not give you full points.
  \end{itemize}

  Example call:%
  \begin{Verbatim}
python3 aes_ecb_cbc.py \ 
  -e -i message.txt \
  -o ciphertext.bin \
  --key 4278b840fb44aaa757c1bf04acbe1a3e \
  --iv 57f02a5c5339daeb0a2908a06ac6393f
  \end{Verbatim}

  Example printed output:%
  \begin{Verbatim}
IV: 57f02a5c5339daeb0a2908a06ac6393f
Key: 4278b840fb44aaa757c1bf04acbe1a3e
  \end{Verbatim}
\end{task}

%Some hints: https://www.pycryptodome.org/src/cipher/aes

%https://www.pycryptodome.org/src/util/util#Crypto.Util.Padding.pad
%https://www.pycryptodome.org/src/util/util#crypto-util-strxor-module

%AES.block_size
%AES.key_size[0]

\begin{task}{ChaCha20 encryption/decryption}
  Homework template file: \texttt{chacha20\_template.py}

  Encrypt the provided file \texttt{message.txt} $5$ times using ChaCha20 with the same key and output the ciphertext bytes into files \texttt{ct-1.bin}, \texttt{ct-2.bin}, \dots.
  NB! The encryptions should be separate, not encapsulated, i.e. not \texttt{Enc(Enc(Enc(...)))}.

  \begin{tcolorbox}[title=Note]
    Make sure that you write the raw ciphertext bytes to the file, and not the bytes of a hex or base64 encoded ciphertext.
  \end{tcolorbox}

  Write the key into a file called \texttt{key.txt} as a hex string, e.g.
  \begin{tcolorbox}[title={\texttt{key.txt}}]
    \texttt{4278b840fb44aa...}
  \end{tcolorbox}

  Write the data needed for the decryption of each file, one hex string per line, into a file named \texttt{decryption.txt}.
  Remember to clear the file at the beginning of each encryption cycle, i.e. there should only be 5 lines in the file.

  Test your program!
  \begin{itemize}
    \item Your program should be able to decrypt all ciphertexts!
    \item Your program should not crash!
  \end{itemize}

  Include the generated files \texttt{ct-\{1-5\}.bin}, \texttt{key.txt}, \texttt{decryption.txt} into your submission ZIP.

  Example call:%
  \begin{Verbatim}
python3 chacha20.py encrypt 4278b840fb44aa... message.txt cts/
  \end{Verbatim}
\end{task}

\end{document}
