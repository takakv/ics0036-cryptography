\graphicspath{ {../../images/} }
\usetikzlibrary{external}

% Needed for ECC graph
\usepackage{pgfplots}

\title{ICS0036 Cryptography}
\subtitle{DL, DH \& ECC}
\date{\today}
\author{Taaniel Kraavi}
\institute%
{%
  \textit{IT College}\\
  \textit{Tallinn University of Technology}
}

\begin{document}
\begin{frame}
  \titlepage
\end{frame}
 
\begin{frame}{Sets}
  A \emph{set} is a collection of distinct (mathematical) objects.
  \begin{itemize}[<+(1)->]
    \item $x\in S$: element $x$ is a member of $S$ (belongs to)
    \item $x\notin S$: element $x$ is not a member of $S$ (does not belong to)
    \item $n = \lvert S\rvert$: $S$ has $n$ elements (cardinality, size)
  \end{itemize}

  \pause
  Sets you must know:
  \begin{itemize}[<+(1)->]
    \item Integers without $0$: $\ZZ\setminus\{0\}=\ZZ^*=\{\dots, -2, -1, 1, 2, 3, \dots\}$
    \item Integers modulo $p$: $\ZZ/p\ZZ = \ZZ_p = \{0, 1, 2, \dots, p-1\}$
    \item Invertible integers modulo $p$: $(\ZZ/p\ZZ)^\times = \ZZ_p^*$ (if $p$ is prime)
    \item $n$-bit bitstrings: $\{0, 1\}^n$
    \item All bitstrings: $\{0, 1\}^*$
  \end{itemize}
\end{frame}

\begin{frame}{Binary operations}
  A \emph{binary operation} is a rule which for any two set elements, gives another set element.
  \begin{itemize}[<+(1)->]
    \item The set is \emph{closed} under the operation
    \item E.g. $\ZZ$ is closed under addition and multiplication
    \item E.g. $\NN$ is not closed under subtraction
  \end{itemize}

  \pause
  A binary operation $\circ$ is a \emph{mapping} of elements:
  \[
    \circ: S \times S \to S
  \]

  \pause
  $\circ$ denotes some arbitrary operation (alternatively $\star, \ast, \diamond$, \dots)
\end{frame}

\begin{frame}{Commutativity \& associativity}
  Let $S$ be a set and $\circ$ a binary operation on $S$.

  \pause
  $\circ$ is \emph{commutative} if for all $a, b\in S$,
  \[
      a \circ b = b \circ a\ .
  \]

  \pause
  $\circ$ is \emph{associative} if for all $a, b, c\in S$,
  \[
    a \circ (b \circ c) = (a \circ b) \circ c\ .
  \]

  \pause
  Questions:
  \begin{itemize}[<+(1)->]
    \item Addition and $\ZZ$?
    \item Subtraction and $\ZZ$?
    \item XOR ($\oplus$) and $\{0,1\}^*$?
  \end{itemize}
\end{frame}

\begin{frame}{Groups}
  A \emph{group} is a set $\GG$ with a binary operation $\circ$ on $\GG$ for which the following properties hold:
  \begin{enumerate}[<+(1)->]
    \item Associativity: for all $a, b, c\in \GG$,
    \[
      a \circ (b \circ c) = (a \circ b) \circ c
    \]
    \item Identity element: there is an element $e\in \GG$ s.t. for all $a\in \GG$,
    \[
      a \circ e = a = e \circ a
    \]
    \item Inverse element: for each $a\in \GG$ there must exist $b\in \GG$ s.t.,
    \[
      a \circ b = e = b \circ a
    \]
  \end{enumerate}
  \vspace*{-1em}
  \pause
  A group is \emph{abelian} (or commutative) if commutativity also holds for all elements.
\end{frame}

\begin{frame}{Examples and bamboozles}
  Questions:
  \begin{columns}[T]
    \begin{column}{0.4\textwidth}
        \begin{itemize}[<+(1)->]
        \item Is $(\NN, +)$ a group?
        \item Is $(\ZZ, +)$ a group?
      \end{itemize}
    \end{column}
    \begin{column}{0.4\textwidth}
      \begin{itemize}[<+(1)->]
        \item Is $(\ZZ^*, +)$ a group?
        \item Is $(\RR^*, \cdot)$ a group?
      \end{itemize}
    \end{column}
  \end{columns}

  \vspace*{1em}

  \pause
  What we know as division on $\RR^*$ is simply the multiplicative inverse:
  \[
    a \cdot b = 1 \iff b = a^{-1} = \frac{1}{a}
  \]

  \pause
  In $\ZZ_p^*$ we define multiplicative inverses through:
  \[
    a\cdot a^{-1} \equiv 1 \pmod{p}.
  \]
  \pause
  The inverse $a^{-1}$ only exists if $\gcd(a,p) = 1$, i.e. when $a$ and $p$ are \emph{coprime}.
\end{frame}

\begin{frame}{Additive and multiplicative notation}
  \begin{center}
  \begin{tabular}{@{}l c c c c@{}}
    \toprule
    Convention & Operation & Identity & \enquote{Powers} & Inverse\\
    \cmidrule{1-5}
    Additive & $x + y$ & $0$ & $nx$ & $-x$\\
    \cmidrule{1-5}
    Multiplicative & $x\cdot y$ or $xy$ & $1$ & $x^n$ & $x^{-1}$\\
    \bottomrule
  \end{tabular}
  \end{center}

  \pause
  E.g. in the multiplicative group $\ZZ_p^*$ with $p$ prime:

  \begin{tabular}{r l}
    \pause $ab$     & $(a \cdot b) \bmod{p}$\\
    \pause $a^b$    & $a^b \bmod{p}$\\
    \pause $a^{-1}$ & $b$ s.t. $ab \equiv 1 \pmod{p}$
  \end{tabular}
\end{frame}

\begin{frame}{Cyclic groups}
  \pause
  Let $\GG$ be a  group, and let $g$ be an element of $\GG$.

  $g$ is a \emph{generator} of $\GG$, denoted $\langle g\rangle$,  if:
  \[
    \GG = \{g^q:q\in\ZZ\} = \{\dots, g^{-1}, 1, g, g^2, g^3, \dots\}
  \]

  \pause
  Example: $\ZZ_7^* = \langle 3\rangle$.
  \pause
  \begin{columns}
    \begin{column}{0.3\textwidth}
      \begin{itemize}
        \item $3^0 \bmod{7} = 1$
        \item $3^1 \bmod{7} = 3$
        \item $3^2 \bmod{7} = 2$
      \end{itemize}
    \end{column}
    \begin{column}{0.3\textwidth}
      \begin{itemize}
        \item $3^3 \bmod{7} = 6$
        \item $3^4 \bmod{7} = 4$
        \item $3^5 \bmod{7} = 5$
      \end{itemize}
    \end{column}
  \end{columns}
\end{frame}

\begin{frame}{Safe primes}
  A \emph{safe} prime is a prime $p$ such that $p = 2q + 1$ with $q$ prime.
  \begin{itemize}[<+(1)->]
    \item Highly important in cryptography
    \item $\ZZ_p^*$ has $p-1$ elements
    \item $\ZZ_p^*$ has a unique subgroup of $q$ elements (subgroup of prime order)
    \item $\ZZ_p^*$ has no \enquote{useful} small subgroups
    \item Security for the discrete logarithm problem (DLP)
  \end{itemize}

  \vspace*{1em}
  \pause
  The \emph{order} of a group, denoted $\lvert\GG\rvert$, is the number of elements in the group.
\end{frame}

\begin{frame}{Discrete logarithm problem (DLP)}
  Setting:
  \begin{itemize}[<+(1)->]
    \item Let $p$ be a (large) safe prime such that $p = 2q + 1$ with $q$ prime.
    \item Fix a generator $g$ and define $\GG = \langle g \rangle = \{g^i :0 \le i < q\}$.
  \end{itemize}

  \vspace*{1em}

  \pause
  The following is then considered \emph{intractable}:
  \[
    \text{Find $x$ s.t. } g^x \equiv y \pmod{p}\ ,
  \]
  where $g, y, p$ are known.

  \pause
  No longer intractable when quantum computers enter the market!
\end{frame}

\begin{frame}{DLP security}
  The DLP exists in every finite cyclic group:
  \pause
  \begin{itemize}
    \item Not guaranteed to be hard in a \enquote{generic} group.
    \item Always research whether the DLP is intractable in a chosen group!
  \end{itemize}

  \vspace*{1em}

  \pause
  Two common group \enquote{types}:
  \begin{itemize}[<+(1)->]
    \item Group of quadratic residues modulo a safe prime
    \begin{itemize}
      \item \enquote{Standard} primes defined in \href{https://datatracker.ietf.org/doc/html/rfc3526}{RFC 3526} ($\ge3072$ bits recommended)
    \end{itemize}
    \item \enquote{Cryptographic} elliptic curve groups (slide \ref{slide:curve-ex})
  \end{itemize}

  \pause
  {\scriptsize\url{https://en.wikipedia.org/wiki/Discrete_logarithm_records}}
\end{frame}

\begin{frame}{Weak DLP group}
  Consider the group $(\ZZ_p^*, +)$.
  \begin{itemize}[<+(1)->]
    \item Not multiplicative, but great for intuition.
    \item DLP (cursed definition): $\log(y) = x \iff xg \equiv y \pmod{p}$.
    \item It follows that $x \equiv yg^{-1} \pmod{p}$.
    \item This is not hard to compute.
  \end{itemize}
\end{frame}

\begin{frame}{Diffie-Hellman key exchange (DHKE)}
  Diffie-Hellman key exchange (DHKE) was the first known key-agreement protocol:
  \begin{itemize}[<+(1)->]
    \item Finite field DH is the \enquote{simplest}: \href{https://datatracker.ietf.org/doc/html/rfc7919}{RFC 7919} (DLP security)
    \item \enquote{Standard} groups in \href{https://datatracker.ietf.org/doc/html/rfc3526}{RFC 3526}
    \item Different variants exist, e.g. ECDH on elliptic curves
    \item Basis for more complex schemes
    \item \href{https://csrc.nist.gov/pubs/sp/800/56/a/r3/final}{NIST SP 800-56A Rev. 3}
  \end{itemize}

  \pause
  Easy to shoot yourself in the foot: a jumbled mess of choices. (my opinion)
\end{frame}

\begin{frame}{Finite field DH}
  \begin{center}
    \begin{tikzpicture}
      \node (bob)   at (0,0)     {\includegraphics[height=80px]{bob}};
      \node (alice) at (8, 0)    {\includegraphics[height=80px]{alice}};
      \node (eve)   at (4, -2.1) {\includegraphics[height=80px]{eve}};

      \node (btop) at (bob.east) [yshift=10px] {};
      \node (atop) at (alice.west) [yshift=10px] {};

      \draw[->,thick] (btop) -- (atop);

      \node (bbot) at (bob.east) [yshift=-10px] {};
      \node (abot) at (alice.west) [yshift=-10px] {};
  
      \draw[<-,thick] (bbot) -- (abot);

      \path let \p1 = (btop) in node (gxt) at (4,\y1) [yshift=7px] {$g^x$};
      \path let \p1 = (bbot) in node (gyt) at (4,\y1) [yshift=7px] {$g^y$};

      \path let \p1 = (btop) in node (circ1) at (2,\y1) {};
      \path let \p1 = (bbot) in node (circ2) at (6,\y1) {};
      
      \draw[->,thick] (circ1.center) |- (eve.west);
      \draw[->,thick] (circ2.center) |- (eve.east);

      \node[anchor=center,draw,circle,fill=white,thick,minimum size = 4px, inner sep=0pt] at (circ1) {};
      \node[anchor=center,draw,circle,fill=white,thick,minimum size = 4px, inner sep=0pt] at (circ2) {};

      \node (gx) at (eve.west) [xshift=-10px, yshift=10px] {$g^x$};
      \node (gy) at (eve.east) [xshift=10px, yshift=10px] {$g^y$};

      \node[anchor=east] (m) at (-0.5, 0.25) {$x\getsu\ZZ_q$};
      \node[anchor=east] (enc) at (-0.5, -0.25) {$g^{xy}\gets\bigl(g^y\bigr)^x$};
      \node[anchor=west] (m) at (8.8, 0.25) {$y\getsu\ZZ_q$};
      \node[anchor=west] (enc) at (8.8, -0.25) {$g^{xy}\gets\bigl(g^x\bigr)^y$};

      \node (enc) at (eve.south) [yshift=-10px] {$g^{xy}$?};
    \end{tikzpicture}
  \end{center}

  \pause
  How to convert the common secret $g^{xy}$ to a valid secret key $\SK\in\{0, 1\}^n$?
\end{frame}

\begin{frame}{DH colours example}
  \begin{center}
    \includegraphics[height=200px]{dhex}
  \end{center}
\end{frame}

\begin{frame}{DH shortcomings}
  \begin{itemize}[<+(1)->]
    \item By default, DHKE is anonymous (no authentication)
    \item Secure against sniffing/eavesdropping, but not against MITM
    \item Easy to pick a weak group (cool \href{https://weakdh.org/imperfect-forward-secrecy-ccs15.pdf}{\textit{article}})
    \item Susceptible\textsuperscript{*} to precomputation attacks: predefined groups/primes
  \end{itemize}
\end{frame}

\begin{frame}{Perfect forward secrecy (PFS)}
  \emph{Ephemeral} Diffie-Hellman (DHE, ECDHE):
  \begin{itemize}[<+->]
    \item Server's long-term key used only for authentication / channel setup
    \item One-time \emph{session} keys generated for each session/transaction
    \item Prevents decryption of past sessions if  long-term key is compromised
    \item Prevents decryption of other sessions if session key is compromised
  \end{itemize}

  \pause
  No absolute protection against future cryptanalysis.
  \begin{itemize}
    \item PFS offers protection only as long as the algorithm remains unbroken.
  \end{itemize}

  \pause
  See also: \emph{\href{https://en.wikipedia.org/wiki/Double_Ratchet_Algorithm}{double ratchet}} for post-compromise confidentiality (e.g. Signal)
\end{frame}

\iffalse
\begin{frame}{Flavours of secrecy}
  \pause
  Forward secrecy (FS):
  \begin{itemize}[<+(1)->]
    \item Subsequent communications are secure post-corruption
    \item Old \enquote{users} cannot read new data (after leaving)
  \end{itemize}

  \pause
  Backward secrecy (future secrecy):
  \begin{itemize}[<+(1)->]
    \item Past communications are secure post-corruption
    \item New \enquote{users} cannot read prior data (before joining)    
  \end{itemize}

  \pause
  Forward secrecy is sometimes called perfect forward secrecy.
  \begin{itemize}
    \item I like to distinguish between the two
    \item Terminology can get confusing
  \end{itemize} 
\end{frame}

\begin{frame}{Perfect forward secrecy}
  The terminology is confusing.

  \begin{itemize}
  \item \emph{Ephemeral} keys: ephemeral DH
  \item Generate short-term \enquote{session} keys (v.s. long-term \emph{static} keys)
  \end{itemize}

  \pause
  No absolute protection against future cryptanalysis.
\end{frame}
\fi

\begin{frame}{ElGamal}
  ElGamal is a probabilistic public-key encryption scheme based on the DLP.

  \pause
  Let $p$ be a safe prime with $p = 2q + 1$, and $\GG$ a subgroup of order $q$ with generator $g$. 
  $(p, q, \GG, g)$ are then the public parameters.

  \pause
  Key generation:
  \begin{itemize}[<+(1)->]
    \item $x \getsu \ZZ_q$
    \item $h \gets g^x \bmod{p}$
    \item $\SK \gets x$
    \item $\PK \gets h$
  \end{itemize}
\end{frame}

\begin{frame}{ElGamal}
  Encryption:
  \begin{itemize}[<+(1)->]
    \item $m\in\GG$
    \item $r\getsu\ZZ_q$
    \item $\ENC_\PK(m; r) = (g^r, m\cdot \PK^r) = (u, v) = c$
  \end{itemize}

  \vspace*{2em}

  \pause
  Decryption:
  \begin{itemize}
    \item $\DEC_\SK(c) = \DEC_\SK((u, v)) = v \cdot u^{-\SK}$
  \end{itemize}
\end{frame}

\begin{frame}{ElGamal's usefulness}
  Nifty properties:
  \begin{itemize}[<+(1)->]
    \item IND-CPA by construction (in a proper group)
    \item ElGamal signature scheme exists
    \item Can be done on elliptic curves for better performance
    \item Multiplicatively homomorphic
    \item Convenient for threshold cryptography
    \item \enquote{Lifted} ElGamal has additive properties (e-voting)
  \end{itemize}

  \pause
  Way too many variations: {\scriptsize\url{https://ibm.github.io/system-security-research-updates/2021/07/20/insecurity-elgamal-pt1}}
\end{frame}

\begin{frame}{Homomorphic encryption}
  Perform meaningful operations on a ciphertext without the secret key.

  \vspace*{1em}

  \pause
  If a cryptosystem is homomorphic for the relation $\circ$, then:
  \begin{itemize}[<+(1)->]
    \item $c_1 \circ c_2 = \ENC_\PK(m_1) \circ \ENC_\PK(m_2) = \ENC_\PK(m_1 \circ m_2) = c_3$
    \item $\DEC_\SK(c_3) = m_1 \circ m_2$
  \end{itemize}

  \vspace*{1em}

  \pause
  Types of homomorphisms:
  \begin{itemize}[<+(1)->]
    \item Partially homomorphic: perform certain operations
    \item Fully homomorphic: perform any computable operation
  \end{itemize}
\end{frame}

\begin{frame}{Homomorphisms of ElGamal}
  \pause
  Multiplicative homomorphism:
  \begin{itemize}
    \item $\ENC_\PK(m_1; r_1) \cdot \ENC_\PK(m_2; r_2) = \ENC_\PK(m_1m_2; r_1 + r_2)$
  \end{itemize}

  \vspace*{1em}

  \pause
  Re-randomisation:
  \begin{itemize}
    \item $\ENC_\PK(m_1; r_1) \cdot \ENC_\PK(1; r_2) = \ENC_\PK(m_1; r_1 + r_2)$
  \end{itemize}
\end{frame}

\begin{frame}{\enquote{Lifted} ElGamal}
  Additive in the exponent:
  \begin{itemize}[<+(1)->]
    \item Encrypt $g^m$ instead of $m$: $\ENC^*_\PK(m; r) = \bigl(g^r, g^m \cdot \PK^r\bigr)$
    \item $\ENC^*_\PK(m_1; r_1) \cdot \ENC^*_\PK(m_2; r_2) = \ENC^*_\PK(m_1 + m_2; r_1 + r_2)$
    \item Recover $m$ by full inspection (message space must be small)
    \begin{itemize}
      \item Essentially DL brute-force (pre-computable)
    \end{itemize}
  \end{itemize}
\end{frame}

\begin{frame}{Elliptic curve cryptography}
  $y^2 = x^3 + ax + b$
  \vspace*{-2em}
  \begin{center}
    \begin{tikzpicture}
    \begin{axis}[
            xmin=-5,
            xmax=5,
            ymin=-5,
            ymax=5,
            ticks=none,
            scale only axis,
            axis lines=middle,
            domain=-2.279018:3,      
            samples=201,
            smooth,   
            clip=false,
            % use same unit vectors on the axis
            axis equal image=true,
            axis line style=thick
        ]
    
    \addplot[blue,thick] {sqrt(x^3-3*x+5)} node[right] {};
    \addplot[blue,thick] {-sqrt(x^3-3*x+5)};
    \end{axis}
    \end{tikzpicture}
  \end{center}
\end{frame}

\begin{frame}{ECDL}
  Elliptic curve discrete logarithm problem:
  \begin{center}
    find $x$ given $\underbrace{P + P + \dots + P}_{x \text{ times}} = x\cdot P = T$
  \end{center}
\end{frame}

\iffalse
\begin{frame}{Elliptic curve cryptography}
  \begin{tikzpicture}
    \begin{axis}[
            xmin=-4,
            xmax=5,
            ymin=-5,
            ymax=5,
            xlabel={$x$},
            ylabel={$y$},
            scale only axis,
            axis lines=middle,
            domain=-2.279018:3,      
            samples=201,
            smooth,   
            clip=false,
            % use same unit vectors on the axis
            axis equal image=true,
        ]
    
    \addplot[blue] {sqrt(x^3-3*x+5)} node[right] {$E$};
    \addplot[blue] {-sqrt(x^3-3*x+5)};
    \addplot[red] {2.621+0.251*(x+1.2)};
    
    
    %\begin{scriptsize}
        %== P ==
        \draw [fill=black] (axis cs:-1.2,2.6) circle (2pt);
        \draw[color=black] (axis cs:-1.4,2.7) node [left]{$P$};
    %\end{scriptsize}
    \end{axis}
    \end{tikzpicture}
\end{frame}
\fi

\begin{frame}{Why elliptic curves}
  Equivalent security with better performance
  \begin{itemize}[<+(1)->]
    \item Large key-sizes for integer factorisation/DLP based trapdoors
    \item Best known algorithms against EC trapdoors are less efficient
    \item More efficient computations and optimisations
    \item The Internet loves them (TLS)
  \end{itemize}
\end{frame}

\begin{frame}{Why not elliptic curves}
  Complex structure
  \begin{itemize}[<+(1)->]
    \item Higher bar of expertise
    \item Supporting PK-encryption is sus (data encoding \enquote{problem})
    \item Questionable parameter generation: are there backdoors?
    \item Different side-channel concerns
    \item Patents, patents, and patents
  \end{itemize}
\end{frame}

\begin{frame}{DualEC\_DRBG}
  Did the NSA backdoor the DualEC\_DRBG ECC-based CSPRNG?
  \begin{itemize}[<+(1)->]
    \item Standardised by NIST
    \item Pushed and promoted by the NSA
    \item Thankfully retired in 2014
  \end{itemize}

  \pause
  Some links:
  \begin{itemize}
    \item {\scriptsize\url{https://blog.cryptographyengineering.com/2015/01/14/hopefully-last-post-ill-ever-write-on/}}
    \item {\scriptsize\url{https://www.schneier.com/blog/archives/2007/11/the_strange_sto.html}}
    \item {\scriptsize\url{https://youtu.be/nybVFJVXbww?si=_fn6J-TD7HnvfPtA}}
  \end{itemize}
\end{frame}

\begin{frame}{Well-known curves}
  \label{slide:curve-ex}
  Some popular curves include:
  \begin{itemize}[<+(1)->]
    \item \textbf{Curve25519} (D. J. Bernstein): good default choice
    \begin{itemize}
      \item Ed25519 EdDSA signatures
      \item X25519 DH over Curve25519
    \end{itemize}
    \item P-256, \textbf{P-384}, P-521 (NIST): FIPS-approved
    \begin{itemize}
      \item secp256r1, secp384r1, secp521r1 
    \end{itemize}
    \item Curve448: if Curve25519 is not enough
    \item \href{https://hackmd.io/@benjaminion/bls12-381}{BLS12-381}: for fancy crypto (e.g. ZK-SNARKs)
    \item Brainpool Curves (\href{https://datatracker.ietf.org/doc/html/rfc5639}{RFC 5639}): German compliance or NIST paranoia
  \end{itemize}
\end{frame}

\begin{frame}{Resources}
  Some resources:
  \begin{itemize}[<+(1)->]
    \item NIST-approved curves for key establishment: \href{https://csrc.nist.gov/pubs/sp/800/186/final}{\textit{NIST SP 800-186}}
    \item NIST-approved curves for digital signatures: \href{https://csrc.nist.gov/pubs/fips/186-5/final}{\textit{FIPS 186-5 (DSS)}}
    \item Bernstein's \href{https://safecurves.cr.yp.to}{\emph{SafeCurves}} guide (arguably biased and a bit outdated)
    \item \href{https://www.keylength.com/en/}{\textit{keylength.com}}
    \item \href{https://cryptobook.nakov.com/asymmetric-key-ciphers/elliptic-curve-cryptography-ecc}{\textit{Nakov's ECC for developers}} (a high-level explanation for devs)
    \item Jan Jancar's \href{https://neuromancer.sk/std/}{\textit{Standard curve database}}
  \end{itemize}

  \pause
  A nice \href{https://crypto.stackexchange.com/questions/93519/why-are-nist-curves-still-used}{\textit{post}} on Crypto Stack Exchange about NIST curves.
\end{frame}

\end{document}
