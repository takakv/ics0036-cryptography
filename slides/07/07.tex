\graphicspath{ {../../images/} }
\usetikzlibrary{external}

\title{ICS0026 Cryptography}
\subtitle{Digital signatures \& PKI}
\date{\today}
\author{Taaniel Kraavi}
\institute%
{%
  \textit{IT College}\\
  \textit{Tallinn University of Technology}
}

\begin{document}
\begin{frame}
  \titlepage
\end{frame}

\begin{frame}{Authenticity of data}
  Authenticity:
  \begin{itemize}[<+(1)->]
    \item Asserts data integrity: not modified in transit
    \item Authenticates the data source: sent by someone we know\textsuperscript{*}
    \item Does not guarantee authorisation!
  \end{itemize}

  \pause
  Symmetric setting:
  \begin{itemize}[<+(1)->]
    \item Message authentication codes (MAC)
    \item Issuing \& verification with the same key
    \item Secret knowledge or possession
    \item Key exchange problem
    \item \emph{Repudiation} problem
  \end{itemize}
\end{frame}

\begin{frame}{Digital signatures}
  Digital signatures:
  \begin{itemize}[<+(1)->]
    \item asymmetric / public-key scheme
    \begin{itemize}
      \item a public \emph{verification key} known to others
      \item a secret \emph{signing key} known to the signer
    \end{itemize}
    \item publicly verifiable: signatures are \emph{transferable}
    \item identification becomes possible\textsuperscript{*}
    \begin{itemize}
      \item keys are personal
      \item verification for a specific signer
      \item signatures are \emph{non-interactive} identification schemes\textsuperscript{*}
    \end{itemize}
    \item \emph{non-repudiation}: you cannot deny your signature
    \item how to tie identity with information?
    \begin{itemize}
      \item legal identification vs tracing
    \end{itemize}
  \end{itemize}
\end{frame}

\begin{frame}{Digital signature scheme}
  A \emph{digital signature scheme} is a triple of PPT algorithms:
  \begin{itemize}[<+(1)->]
    \item $\GEN$ is a \emph{key generation algorithm}: $(\PK, \SK)\gets\GEN$
    \item $\SIGN$ is a \emph{signing algorithm}: $\sigma\gets\SIGN_\SK(m)$
    \item $\VERIF$ is a deterministic \emph{verification algorithm}: $b\gets\VERIF_\PK(m, \sigma)$
  \end{itemize}

  \pause
  Notation and terms:
  \begin{itemize}[<+(1)->]
    \item $\SK$ is the private/secret key also called the \emph{signing key}
    \item $\PK$ is the public key also called the \emph{verification key}
    \item $b$ is the verification result (bit): $1$ meaning valid, $0$ meaning invalid
    \item $\sigma$ is a valid \emph{signature} on $m$ if $b = 1$
  \end{itemize}

  \pause
  The signature scheme is \emph{functional} if $\VERIF_\PK(m, \SIGN_\SK(m)) = 1$.
\end{frame}

\begin{frame}{Digital signature scheme}
  \pause
  \begin{center}
    \begin{tikzpicture}
      \node (alice) at (0,0)  {\includegraphics[height=80px]{alice}};
      \node (bob)   at (8, 0) {\includegraphics[height=80px]{bob}};
      \node (eve)   at (4, 0) {\includegraphics[height=80px]{eve}};

      \draw[<-, thick] (eve.east) -- node[above] {$(m, \sigma$)} (bob.west);
      \draw[<-, thick] (alice.east) -- node[above] {$(\overline{m}, \overline{\sigma}$)} (eve.west);
      
      \node[anchor=east] (gen) at (11, 2.5) {$(\PK,\SK)\gets\GEN$};
      \draw[<-,dashed,thick] (alice.north) |- (gen.west);

      \node (el) at (eve.north) [xshift=-5px] {};

      \path let \p1 = (el) in node (circ) at (\x1,2.5) {};
      \node at (circ.north) [yshift=5px] {$\PK$};

      \node[anchor=west]  at (8.5, 0.25)  {$m\gets\MMM$};
      \node[anchor=west]  at (8.5, -0.25) {$\sigma\gets\SIGN_\SK(m)$};
      \node[anchor=north] at (alice.south) {$\VERIF_\PK(\overline{m},\overline{\sigma})\iseq 1$};

      \draw[->,dashed,thick] (circ) -- (el);
      \node[anchor=center,draw,circle,fill=white,thick,minimum size = 2px, inner sep=0pt] at (circ) {};
    \end{tikzpicture}
  \end{center}

  \pause
  Mallory should not be able to \emph{forge} a valid signature.
\end{frame}

\begin{frame}{Security definitions}
  Attack scenarios
  \begin{itemize}[<+(1)->]
    \item \emph{Key only attack.}
    The adversary has access only to the public key $\PK$.
    \item \emph{Chosen-message attack.}
    Besides having $\PK$, the adversary can query a list of valid signatures $(m_1, \sigma_1), \dots, (m_n, \sigma_n)$.
  \end{itemize}

  \pause
  Attack types
  \begin{itemize}[<+(1)->]
    \item \emph{Universal forgery.}
    Create a valid signature for a prescribed message $m$.
    \item \emph{Existential forgery.}
    Create a valid signature for some message $m$.
  \end{itemize}

  \pause
  A \emph{secure} scheme is secure against the \emph{one-more signature attack}:
  \begin{itemize}[<+(1)->]
    \item chosen-message attack + existential forgery
    \item existentially unforgeable under an adaptive chosen-message attack
  \end{itemize}
\end{frame}

\begin{frame}{How to sign?}
  Hints:
  \begin{itemize}[<+(1)->]
    \item Messages can be large
    \item PK systems do not operate on arbitrary size data
    \item Different messages: different signatures
  \end{itemize}

  \pause
  Common approach:
  \begin{itemize}[<+(1)->]
    \item cryptographic hash: message representative
    \item hash and sign the hash (hash used in the algorithm)
    \item how do you verify? (message not recoverable from the hash)
  \end{itemize}

  \pause
  Other approaches exist
\end{frame}

\begin{frame}{RSA signatures}
  \pause
  PKCS \#1:
  \begin{itemize}[<+(1)->]
    \item \texttt{RSASSA-PKCS1-v1\_5} no known weakness, but lacks formal proofs
    \item \texttt{RSASSA-PSS} (introduced in PKCS \#1 v2.1)
    \item \href{https://datatracker.ietf.org/doc/html/rfc8017}{\texttt{RFC 8017}}
  \end{itemize}

  \vfill

  \pause
  PSS --- Probabilistic Signature Scheme
  \begin{itemize}[<+(1)->]
    \item Renders signatures probabilistic
    \item Has a security analysis: provably reducible to the RSA problem
    \item Parameters must be known before signature verification 
  \end{itemize}
\end{frame}

\begin{frame}{RSA-PSS parameters}
  RSA-PSS uses a mask generation function (MGF) for the padding
  \begin{itemize}[<+(1)->]
    \item Similar to a cryptographic hash function
    \item Variable length output
    \item Specifically MGF1: iterative hashing with counter
    \item MGF1 also used in RSA-OAEP
  \end{itemize}

  \vfill

  \pause
  Hashing in RSA-PSS:
  \begin{itemize}[<+(1)->]
    \item Uses SHA1 by default: if possible, parametrise with a SHA2 variant
    \item If possible, use the same hash function for the message and MGF1
  \end{itemize}
\end{frame}

\begin{frame}{Determinism vs. nondeterminisim}
  \pause
  Unlike for encryption, non-determinism is not crucial per se:
  \begin{itemize}[<+(1)->]
    \item no message to hide
  \end{itemize}

  \pause
  Nonces protect the key in many constructions:
  \begin{itemize}[<+(1)->]
    \item nonce leak might allow secret recovery
    \item nonce reuse might allow secret recovery
    \item conclusion: nonces must be unique and hard to find
    \item (secrecy over nondeterminism)
  \end{itemize}

  \pause
  Poor randomness is catastrophic in many schemes!
  \begin{itemize}[<+(1)->]
    \item Good randomness helps with uniqueness and secrecy
  \end{itemize}
\end{frame}

\begin{frame}{Randomness pros and cons}
  \pause
  Issues with randomness:
  \begin{itemize}[<+(1)->]
    \item Cryptographic randomness required (real-world security issue)
    \item Covert channel attacks (steganography, e.g. physical crypto wallets)
  \end{itemize}

  \pause
  Pros of randomness:
  \begin{itemize}[<+(1)->]
    \item Typically less reliance on the security of primitives
    \item Easier to prove secure
    \item `More robust' against partial breakage
  \end{itemize}

  \pause
  Deterministic signatures more and more common (e.g. blockchains).
  \begin{itemize}[<+(1)->]
    \item Proofs and standards for deterministic schemes
  \end{itemize}
\end{frame}

\begin{frame}{DSA \& ECDSA}
  \pause
  Digital Signature Algorithm (DSA):
  \begin{itemize}[<+(1)->]
    \item DL-based (variant of Schnorr \& ElGamal)
    \item Part of DSS versions 1--4, removed in version 5 (2023)
    \item ECDSA: Elliptic Curve DSA
    \item \href{https://datatracker.ietf.org/doc/html/rfc6979}{\texttt{RFC 6979}}: deterministic DSA \& ECDSA
  \end{itemize}

  \pause
  Sony 2013 PS3 attack:
  \begin{itemize}
    \item ECDSA software signing-key recovered
    \item Caused by nonce-reuse (Sony implementation issue)
  \end{itemize}

  \pause
  Do not use DSA! Do not default to ECDSA.
\end{frame}

\begin{frame}{EdDSA}
  Edwards-curve Digital Signature Algorithm (\href{https://datatracker.ietf.org/doc/html/rfc8032}{\texttt{RFC 8032}}):
  \begin{itemize}[<+(1)->]
    \item Not to be confused with DSA \& ECDSA: different algorithm entirely
    \item Designed by \alert{Bernstein} et al.
    \item Based on twisted Edwards Curves (a model of ECs)
    \item High performance
    \item Designed with side-channel safety in mind
    \item Ed25519 (Curve25519 with SHA512) \& Ed448 (Curve448 with SHAKE256)
    \item Deterministically chosen nonce
  \end{itemize}

  \pause
  Unless you have a good reason not to, use EdDSA!
\end{frame}

\begin{frame}{DSS}
  Digital Signature Standard (\href{https://csrc.nist.gov/pubs/fips/186-5/final}{FIPS 168-5}):
  \begin{itemize}[<+(1)->]
    \item No longer allows DSA for issuing signatures
    \begin{itemize}
      \item Prior signature verification still allowed
    \end{itemize}
    \item ECDSA is allowed
    \begin{itemize}
      \item Forbidden to use it for non-signing purposes (e.g. key establishment)
      \item Deterministic \& nondeterministic variants
    \end{itemize}
    \item Deterministic EdDSA is allowed
    \begin{itemize}
      \item Both Ed25519 and Ed448 variants
      \item Variants differ a bit from the RFC
    \end{itemize}
    \item RSA signatures as defined in PKCS \#1 v2.2 are allowed
    \begin{itemize}
      \item Additional restrictions apply
      \item No textbook RSA!
    \end{itemize}
  \end{itemize}
\end{frame}

\begin{frame}{Long term signatures}
  How to archive digitally signed documents?
  \begin{itemize}[<+(1)->]
    \item In 20 years, how to check if a signature was valid today?
    \item Long-term validation (LTV)
    \item We will see formats and legality next week
  \end{itemize}

  \pause
  Re-signing over a document:
  \begin{itemize}[<+(1)->]
    \item Sign it with a new scheme before the old one is broken
    \item Timestamp the new signature for temporal order
    \item Extend your signature lifetime
  \end{itemize}

  \pause
  We will see timestamping also next week.
\end{frame}

\begin{frame}{PQ signatures}
  Post-quantum safe signature schemes chosen by NIST:
  \begin{itemize}[<+(1)->]
    \item CRYSTALS-Dilithium (lattices)
    \item Falcon (lattices)
    \item SPHINCS+ (hashes)
  \end{itemize}

  \vspace*{0.5em}

  \pause
  `Hybrid mode': combination of PQ and pre-quantum schemes.
  \begin{itemize}[<+(1)->]
    \item separate signatures vs. composite mode?
  \end{itemize}

  \vspace*{0.5em}

  \pause
  PQ schemes are not yet widely used
  \begin{itemize}
    \item long keys and signatures
    \item slower than classical counterparts
  \end{itemize}
\end{frame}

\begin{frame}{Hash-based signature schemes}
  \pause
  Cryptographic hash functions are usually PQ resistant.
  \begin{itemize}
    \item Can we build a signature scheme out of them?
  \end{itemize}

  \vspace*{1em}

  \pause
  At the source: Lamport's one-time signatures
  \begin{itemize}[<+(1)->]
    \item Based on one-way functions (not only CHFs)
    \item One key per signature
    \item Optimisations exist, e.g. with PQ CSPRNGs
    \item Keys can be combined using structures, e.g. Merkle trees
  \end{itemize}
\end{frame}

\begin{frame}{Cool constructions}
  \pause
  Group signatures:
  \begin{itemize}[<+(1)->]
    \item Anonymously sign on behalf of a group
    \item The public cannot learn who this is
    \item The group `manager' can (unforgeably) trace and de-anonymise a signer
    \item Signatures are unlinkable (multiple signatures by the same issuer?)
  \end{itemize}

  \pause
  Ring signatures:
  \begin{itemize}[<+(1)->]
    \item Group signatures without a manager/trusted setup
    \item Traceable/linkable variants exist
  \end{itemize}

  \pause
  Blind signatures
  \begin{itemize}[<+(1)->]
    \item Sign some data without knowing the exact contents
  \end{itemize}
\end{frame}

\begin{frame}{Cool constructions}
  \pause
  Threshold signatures:
  \begin{itemize}[<+(1)->]
    \item Can only produce a valid signature in a quorum
    \item We will cover threshold cryptography in more depth in some weeks
  \end{itemize}

  \pause
  Multisignatures:
  \begin{itemize}[<+(1)->]
    \item Combine multiple signatures into a single succinct signature
    \item All signatories are evident from the multisignature
  \end{itemize}

  \pause
  Identity escrow:
  \begin{itemize}[<+(1)->]
    \item Party $A$ gives information to party $B$
    \item This information does not allow $B$ to identify $A$
    \item A 3\textsuperscript{rd} party $C$ can use this info to identify $A$
    \item $B$ receives a guarantee that $C$ can identify $A$
  \end{itemize}
\end{frame}

\begin{frame}{Trust models}
  How do we establish trust?
  \begin{itemize}[<+(1)->]
    \item Hierarchical trust (CAs)
    \item Web of trust (WoT)
    \item Trust on First Use (TOFU)
    \item Dual control model
    \item \dots
  \end{itemize}

  \pause
  Someone somewhere has to trust something sometime.
\end{frame}

\begin{frame}{Digital certificates}
  A \emph{certificate} binds keys to an entity's attributes (e.g. name, email).
  \begin{itemize}[<+(1)->]
    \item An entity lists their attributes and public key on a certificate
    \item A third party verifies the info and signs the certificate
    \begin{itemize}
      \item The party confirms the attributes somehow
      \item The party verifies secret key ownership of the entity
    \end{itemize}
    \item Self-signing is possible
  \end{itemize}

  \pause
  Ideally, the 3\textsuperscript{rd} party is like a digital notary.
  \begin{itemize}[<+(1)->]
    \item You simply trust the notary
    \item Trust the notary $\to$ trust their approved certificate\textsuperscript{*}
    \item Verification requirements vary (national eID vs. git commit)
  \end{itemize}
\end{frame}

\begin{frame}{Public Key Infrastructure (PKI)}
  A public key infrastructure consists of a:
  \begin{itemize}[<+(1)->]
    \item certificate authority (CA): stores, issues \& signs certificates
    \item registration authority (RA): verifies the identity of certificate requesters
    \item central directory: indexing of keys
  \end{itemize}

  \pause
  Entities need not all be separate, e.g. CA \& RA could be one.
  \begin{itemize}
    \item The term \emph{trusted third party} (TTP) often used
  \end{itemize}
\end{frame}

\begin{frame}{PKI (extended)}
  \pause
  There is also usually a:
  \begin{itemize}[<+(1)->]
    \item certificate policy: defines parties' roles and procedures
    \item certificate management system
  \end{itemize}

  \pause
  These are less critical for understanding how the PKI works.
\end{frame}

\begin{frame}{Root CAs}
  \begin{block}{Terminology}
    Think of a CA as a (key, certificate) pair, and not as a company.
  \end{block}

  \pause
  Root CAs are trust anchors:
  \begin{itemize}[<+(1)->]
    \item Self-sign their certificates: root certificates
    \item Root certificates are bundled with the OS, with browsers
    \item Certify lower level CAs, called \emph{intermediate} CAs
    \begin{itemize}
      \item Intermediate certificates
      \item Typically the intermediate CA is the same company
    \end{itemize}
  \end{itemize}

  \pause
  New root certificates can manually be added to a device's root certificate store.
\end{frame}

\begin{frame}{Chain of Trust}
  A chain of trust is derived from a trust anchor:
  \begin{itemize}[<+(1)->]
    \item Root CAs certify and assign powers to intermediate CAs
    \begin{itemize}
      \item Intermediate CAs might be able to delegate to even further CAs etc.
      \item Multi-domain (company) relationships defined in \href{https://datatracker.ietf.org/doc/html/rfc5217}{RFC 5217}
    \end{itemize}
    \item End users are issued \emph{end entity} certificates by a CA
    \begin{itemize}
      \item Cannot be used to further certify certificates
      \item Not certified directly by a root certificate
    \end{itemize}
    \item Certificate chain:
    \begin{center}
      root cert $\to$ intermediate cert(s) $\to$ end entity cert
    \end{center}
    \vspace*{1em}
    \item To verify a certificate, recursively verify until the root
    \begin{itemize}[<+(1)->]
      \item How much to verify?
    \end{itemize}
  \end{itemize}
\end{frame}

\begin{frame}{Web of Trust}
  An alternative to the PKI chain of trust:
  \begin{itemize}[<+(1)->]
    \item Popular with PGP
    \item Not centralised unlike CA-based systems (PKIs in general)
    \item LinkedIn for public keys
    \item Upload your keys to key-servers on the web (no deletion possible usually)
    \item You \emph{vet} a user by signing their keys
    \item Others who trust you can then trust the keys you have signed
    \item You can assign a trust level to keys
  \end{itemize}

  \pause
  It's a mess, especially the key management.
\end{frame}

\end{document}
