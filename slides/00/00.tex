\title{ICS0036 Cryptography}
\subtitle{Course organisation}
\date{\today}
\author{Taaniel Kraavi}
\institute%
{%
  \textit{IT College}\\
  \textit{Tallinn University of Technology}
}

\begin{document}
\begin{frame}
  \titlepage
\end{frame}

\begin{frame}
  \frametitle{Course aim}

  \begin{columns}[T]
    \begin{column}{.5\textwidth}
      \textbf{Prerequisites:}
      \begin{itemize}[<+->]
        \pause
        \item Know basic programming
        \begin{itemize}
          \item C \& python3
        \end{itemize}
        \item Notions in:
        \begin{itemize}
          \item Discrete mathematics
          \item Binary data representation
          \item Group theory (bonus)
        \end{itemize}
      \end{itemize}
    \end{column}

    \pause
    \begin{column}{.4\textwidth}
      \textbf{Approach:}
      \begin{itemize}[<+->]
        \item Common terminology
        \item Cryptographic primitives as a black-box API
        \item Pitfalls
        \item DYOR: due diligence
      \end{itemize}
    \end{column}
  \end{columns}

  \vspace*{2em}

  \pause
  ITC8240 (master's course) for mathematical cryptography.\\
  \pause
  ITC8290 (master's course) for post-quantum cryptography.
\end{frame}

\begin{frame}
  \frametitle{Schedule}

  \textbf{Lectures}

  Mondays: 12:00--13:30 room ICO-217

  \vspace*{1em}

  \pause
  \textbf{Practices}

  Wednesdays: 14:00--15:30 room ICO-410

  \vspace*{1em}

  \pause
  Attendance \& important dates
  \begin{itemize}[<+(1)->]
    \item Attend at least 8 lectures \& 8 practices: laxer exam pass requirements
    \item October 29: in-person midterm
    \item November 12, 19, 26: group presentations
    \item December 17: first written exam attempt
  \end{itemize}
\end{frame}

\begin{frame}
  \frametitle{Grading}

  \pause
  \textbf{Grade distribution}
  \begin{itemize}
    \item 40\% --- homework assignments
    \item 10\% --- group presentation
    \item 50\% --- final exam
  \end{itemize}

  \vspace*{1em}

  \pause
  \begin{tabular}{cccccc}
    $\le 51$p or missed requirements & $[51, 60]$ & $[61, 70]$ & $[71, 80]$ & $[81, 90]$ & $[91, 100]$\\
    \midrule
    0 & 1 & 2 & 3 & 4 & 5
  \end{tabular}
\end{frame}

\begin{frame}
  \frametitle{Grading}

  \textbf{Homework}
  \begin{itemize}[<+(1)->]
    \item Up to 6 mandatory assignments
    \item 2 weeks to complete
    \item Theoretical part (research) \& practical part (programming)
    \item Best-effort:
    \begin{itemize}
      \item Important to try
      \item You can resubmit an improved version
      \item Assignment grade: original grade + 50\% of fixes grade
    \end{itemize}
  \end{itemize}

  \pause
  Exam requirement: $\ge 20$ points from home tasks.
\end{frame}

\begin{frame}
  \frametitle{Grading}

  \pause
  \textbf{Midterm}
  \begin{itemize}
    \item Mandatory to attend
    \item Reality check for your knowledge
    \item No effect on final grade
  \end{itemize}

  \vspace*{1em}

  \pause
  \textbf{Group presentation}
  \begin{itemize}
    \item In-person
    \item Groups of 2 or 3
    \item Topics TBD
    \item Worth 10 points, may vary by member
  \end{itemize}
\end{frame}

\begin{frame}
  \frametitle{Grading}

  \textbf{Exam}
  \begin{itemize}[<+(1)->]
    \item 1h 30min
    \item Closed book \& written
    \begin{itemize}
      \item You can bring a one-sided A4 of \emph{handwritten} notes
      \item Alternatively: 20min oral exam (max 10 students)
    \end{itemize}
    \item Must have attended the midterm + oral presentation
    \item Must have at least 20 points from homework tasks
    \item Must pass each exam section to pass the exam
  \end{itemize}
\end{frame}

\begin{frame}
  \frametitle{Grading}

  \textbf{Summary}
  \begin{itemize}[<+(1)->]
    \item Home tasks (40p, of which 20p are mandatory)
    \item Midterm (ungraded, but mandatory)
    \item Group presentation (10p) 
    \item 90min written exam (50p) with minimal pass thresholds
  \end{itemize}

  \vspace*{1em}

  \pause
  Not hard to pass, but requires consistent focus.
  \begin{itemize}
    \item LLMs allowed as a \emph{helper} for homework (but not recommended)
    \item LLMs without understanding the subject: you will likely fail the exam
  \end{itemize}
\end{frame}

\begin{frame}
  \frametitle{Syllabus}

  \begin{columns}[t]
    \begin{column}{.5\textwidth}
      \begin{enumerate}
        \item Introduction
        \item Symmetric cryptography
        \item Symmetric cryptography (cont.)
        \item Public-key cryptography
        \item Public-key cryptography (cont.)
        \item Hash functions \& MACs
        \item Digital signatures \& certificates
        \item Public-key infrastructure \& eIDAS
      \end{enumerate}
    \end{column}

    \begin{column}{.5\textwidth}
      \begin{enumerate}\setcounter{enumi}{8}
        \item TLS, HTTPS, SSH
        \item Secure programming \& hardware
        \item Authentication \& commitments
        \item Zero-knowledge proofs
        \item Secret sharing \& MPC
        \item Internet voting
        \item Post-quantum cryptography
        \item Review
      \end{enumerate}
    \end{column}
  \end{columns}

  \vspace*{2em}

  \pause
  Topics might not be cleanly split by week \& the order might change.
\end{frame}

\begin{frame}
  \frametitle{Reading materials}

  The book \enquote{The Joy of Cryptography} by Mike Rosulek is freely available at \url{https://joyofcryptography.com} and covers many of the topics in this course.

  \vspace*{1em}

  The book is more mathematical than this course, but can hence provide you with valuable insight and a stronger understanding of the same concepts. 
\end{frame}

\end{document}
